\documentclass[12pt]{article}
\usepackage[english]{babel}
\usepackage{natbib}
\usepackage{url}
\usepackage[utf8x]{inputenc}
\usepackage{amsmath}
\usepackage{graphicx}
\graphicspath{{img/}}
\usepackage{parskip}
\usepackage{fancyhdr}
\usepackage{vmargin}
\setmarginsrb{3 cm}{2.5 cm}{3 cm}{2.5 cm}{1 cm}{1.5 cm}{1 cm}{1.5 cm}


\makeatletter
\let\thetitle\@title

\let\thedate\@date
\makeatother

\pagestyle{fancy}
\fancyhf{}
\rhead{\theauthor}
\lhead{\thetitle}
\cfoot{\thepage}

\begin{document}

%%%%%%%%%%%%%%%%%%%%%%%%%%%%%%%%%%%%%%%%%%%%%%%%%%%%%%%%%%%%%%%%%%%%%%%%%%%%%%%%%%%%%%%%%

\begin{titlepage}
	\centering
    
    \includegraphics[scale = 0.75]{Logo.png} \\[1.0 cm]
    
    \vspace{4 cm}
    
    {\LARGE Experiment report} \\[0.2 cm]
    
    \vspace{1 cm}
    
    \rule{\linewidth}{0.2 mm} \\[0.4 cm]
    
	\textsc{\LARGE Ethereum vs. Hyperledger}\\[0.2 cm]
	\textsc{\lARGE Comparison of throughput for non-conflicting transactions} \\[0.2 cm]
	\rule{\linewidth}{0.2 mm} \\[0.4 cm]
	
	{\huge \bfseries \thetitle}\\
	
	\vspace{2 cm}
	
	\begin{minipage}{0.4\textwidth}
		
		\begin{flushleft} 
			\emph{Authors:} \\
			Jacek Janczura \linebreak
			Igor Molcean \linebreak
			Julian Valentino Weigel \linebreak
			Kim Janik Jasun Herter
		\end{flushleft}
	\end{minipage}\\[2 cm]
	

 
	\vfill
	
\end{titlepage}

%%%%%%%%%%%%%%%%%%%%%%%%%%%%%%%%%%%%%%%%%%%%%%%%%%%%%%%%%%%%%%%%%%%%%%%%%%%%%%%%%%%%%%%%%

\tableofcontents
\pagebreak

%%%%%%%%%%%%%%%%%%%%%%%%%%%%%%%%%%%%%%%%%%%%%%%%%%%%%%%%%%%%%%%%%%%%%%%%%%%%%%%%%%%%%%%%%

\begin{abstract}
In this experiment, we compared Ethereum and Hyperledger private networks with regard to throughput for non-conflicting transactions, i.e. transactions that do not cause double-spending. We proposed two mathematical definitions of throughput and used them to compare both systems. We set up the networks in similar conditions and provided them with synthetically generated workload. We used custom tools to collect relevant data from systems’ logs. Finally, we came to the conclusion that Ethereum, generally, shows more promising results. However, further research will be needed for more profound analysis.
\end{abstract}

\newpage
\section{INTRODUCTION}

Blockchain is a relatively new concept. It represents a distributed, uncentralized and immutable data storage where every single change can be seen by anyone and the whole amount of data is stored at every node that participates in the network. This makes blockchain quite different from traditional databases that try to restrict access as far as possible and require centralized coordinator for replication. Unlike many distributed storages, blockchain does not use redundancy to increase performance but rather to improve resilience of the system. For the first time, the term was used in context of Bitcoin in 2008 \cite{bitcoin}. This became the reason why blockchain is often associated with cryptocurrencies but potencial of this technology is actually much higher.

One of the big problems that prevents blockchain-based technologies from entering our everyday life is limited throughput. The exact definition of it will be given in subsection \ref{throughput}, for now we will just say it is the number of operations that can be performed on a system per time unit. For example, limited block size in case of Bitcoin, caused the so-called "Bitcoin scalability problem" which resulted in intense research \cite{bitcoin_scaling} and numerous proposals on how to increase throughput of the system. Being able to perform operations fast, is important for a system that is used by big number of people worldwide, this is why throughput becomes so critical in context of blockchain systems.

The goal of our experiment is to measure the throughput for two widely used blockchain systems, Ethereum and Hyperledger, in order to learn how effective their underlying algorithms process big numbers of submitted transactions in short periods of time.

\subsection{Transaction throughput} \label{throughput}
In out experiment, we defined throughput as average number of transactions in a block divided by the block frequency:

$$Throughput = \frac{1/N \sum_{b=1}^{N} T_b}{F}$$

where:
\begin{itemize}
    \item $N$ is total number of transactions submitted
    \item $T_b$ is number of transactions contained in block $b$
    \item $F$ is block frequency
\end{itemize}

Other definitions can also be found in literature. We decided to compare one of them with ours. It is formulated as follows: transaction throughput is the rate at which valid transactions are committed by the blockchain System Under Test (further SUT) in a defined time period \cite{hyperledger_paper}. This definition may be presented as the following formula:

$$Throughput_{alt} = \frac{N}{t_{b_N} - t_{T_0}}$$

where:
\begin{itemize}
    \item $N$ is total number of submitted transactions submitted
    \item $t_{b_N}$ is the commit time of the last block
    \item $t_{T_0}$ is the submission time of the initial transaction
\end{itemize}


\section{METHOD}

\section{RESULTS}

\section{FUTURE SCOPE}

\section{CONCLUSION}


\newpage
\bibliographystyle{plain}
\bibliography{biblist}

\end{document}
