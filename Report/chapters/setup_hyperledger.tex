\textit{Author: Julian} \\
For Hyperledger we used the Burrow framework. Burrow was selected, because it uses the same EVM as Etherium, 
therefore the same language for contracts could be used. There are 2 EC2 instances running. One for the Blockchain system
running with one registered validator and one normal participant. The second instance is used to send the
transactions to the first node. \\
To create a usable configurate for Burrow to run, the following code was used.
\begin{lstlisting}[frame=single]
    burrow spec -p1 -f1 | burrow configure -s- > burrow.toml
\end{lstlisting}
The key pairs for one validator and one participants were generated.
The automatic generated config files listening addresses had to be changed from \textit{127.0.0.1:port} to \textit{0.0.0.0:port}
for it to work on two different machines. To avaid double spending, the initial funds were set to a high amount and the transaction
amount during the testing phase switched to one.