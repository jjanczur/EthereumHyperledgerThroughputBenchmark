\subsection{Main concepts}

\textbf{Throughput} is the maximum rate at which transactions can be successfully processed by the blockchain, i.e. end up in a block of the canonical chain. Not to be confused with \textbf{Bandwidth}, which is the maximum possible rate at which transactions can be processed theoretically. Both are measured in transactions per second (Tx/s).

\textbf{Proof-of-Authority} is an algorithm where only authorized nodes, called sealers or signers, can add blocks to chain. To make sure a malicious node can’t do harm to the network, any signer can sign at most one of a number $\lfloor S/2 \rfloor + 1$ of consecutive blocks (where $S$ is the number of sealers). PoA algorithm used in Ethereum is called Clique \cite{clique}.

\textbf{Proof-of-Stake} is a consensus algorithm where decisions whether to accept a transaction or not are made based on weighted random choice. It is calculated by participation
duration in combination with the size of the stake. The proof of stake algorithm is fully byzantine fault tolerant \cite{pos}. A high transaction throughput is
reached via the used Tendermint consensus engine \cite{burrow_git}.
