\subsection{Hyperledger}

\textit{Author: Julian} \\
In Hyperledger Burrow the throughput was measured using the $Throughput_{alt}$ method mentioned in 2.1.
In \ref{table:4} the minimum, maximum and average time of a transaction is visualized. For 10 up to 1000 transactions
the minimum and average time of one transaction is below 1 second. The maximum time for more than 1000 transactions
drastically increases from 25 seconds to around 1000 seconds for 10000 transactions.

\begin{table}[!h]
\centering
\begin{tabular}{l|l|l|l|}
\hline
\multicolumn{1}{|l|}{\textbf{tx}} & \textbf{Min {[}s{]}} & \textbf{Max {[}s{]}} & \textbf{Avg {[}s{]}} \\ \hline
\multicolumn{1}{|l|}{\textit{10}} & \textit{0.436} & \textit{0.632} & \textit{0.530} \\ \hline
\multicolumn{1}{|l|}{\textit{100}} & \textit{0.379} & \textit{0.644} & \textit{0.475} \\ \hline
\multicolumn{1}{|l|}{\textit{1000}} & \textit{0.387} & \textit{25.557} & \textit{0.572} \\ \hline
\multicolumn{1}{|l|}{\textit{10000}} & \textit{0.359} & \textit{999.583} & \textit{3.457} \\ \hline 
\end{tabular}
\caption{Burrow: Min, Avg and Max execution time of a transaction}
\label{table:4}
\end{table}

The throughput for the test with 10 transactions as displayed in figure \ref{fig:burrowtxs} upper left graph shows a more or less
stable curve with only few ups and downs and not much variation. The result for 100 (figure \ref{fig:burrowtxs} upper right)
transactions has a similar outcome. Starting the benchmark with 1000 (figure \ref{fig:burrowtxs} lower left) or even 10000
(figure \ref{fig:burrowtxs} lower right) transactions, the variance of some of these measurements is extremly high. In
the last test, the effect even intensifies with increased time passed. \\[1cm]

\begin{minipage}{\linewidth}
    \makebox[\linewidth]{
    \includegraphics[width=1.00\textwidth]{img/HyperAllInOne.png}}
   \captionof{figure}{Burrow: 10 transactions and their execution time}\label{fig:burrowtxs}
\end{minipage}

%%%Maybe switch the following part to conclusion
The increase in the time needed to execute one transaction with a higher total number of transactions sent is part of the
limitations of the used EC2 instance. The thread pool used to send multiple transactions at once were to many for the engine
to handle and in the end the node ran out of memory to use. This effect could be weakened by decreasing the thread pool size
but would also result in a higher total execution time. A test was done with 1000 transaction and a pool size of 10 and 150.
Where the total execution time of the first one was around 20 minutes, the second one was below 4 minutes.


