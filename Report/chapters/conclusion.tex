\section{Conclusion}
\textit{Author: Kim}\\
For further research it would be necessary to investigate Hyperledger more intense. This is especially because of the lack of time that resulted in small numbers of transaction in the Hyperledger setup. To improve the results in that matter the number of transactions has to be increased.\\
In addition to that, Hyperledger was used with the Burrow framework only. That is why in further experiments the variety of frameworks that were tested has to be increased.\\
\\
In our experiment the goal was to compare the two private blockchain networks Ethereum and Hyperledger with regard to its throughput for non-conflicting transactions.\\
To achieve this we proposed two mathematical definitions of throughput and used these to compare the two networks. For data collection we deployed several AWS EC2 instances and ran the networks as well as clients to send transactions on these. In both cases we could collect the data in a log-File and then had to run our own scripts to collect the necessary information. After collecting the data we were able to calculate the transaction throughputs for both, Ethereum and Hyperledger.\\
Comparing the results from Ethereum and Hyperledger leads to the conclusion, that Ethereum seems to be more efficient. \\
Nonetheless, this needs further research, especially to test it with higher number of transactions for Hyperledger.